\subsection{Invertierender Transistor
}\label{subsec:topo-inv-transistor}

Diese Schaltung stellt eine Emitterschaltung dar. Sie nutzt einen NPN Transistor als Schalter um die Spannungswandlung vorzunehmen.
Ein Vorteil ist das sie die Spannung sowohl nieder als auch hoch setzten kann. Ein Nachteil ist, dass sie den Logik Pegel invertiert. 
\begin{figure}
  \begin{circuitikz}
    \draw (0,0) node[npn](Q1) {2N2222}
    (Q1.base) node[anchor=east] {}
    (Q1.collector) node[anchor=south] {}
    (Q1.emitter) node[anchor=north] {};
    \draw (Q1.collector) to[R, l2=$R_1$ and \SI {10}{k\ohm}] ++(0,2)
    node[vcc](){\SI{3.3}{V}};
    \draw (Q1.collector) to[short, -o] ++(2,0) node[right]{$output$};
    \draw (Q1.emitter) node[ground] {} ++(0,-1);
    \draw (Q1.base) to[R, -o, l2=$R_2$ and \SI {10}{k\ohm}] ++(-2,0)
    node[left]{$input$};
  \end{circuitikz}
  \caption{inv. Transistor}
  \label{fig:spannungsteiler}
\end{figure}

Der Einganswiderstand berechnet sich den Basiswiederstand und $R_{be}$:

\begin{math}
  R_i = R_2 + R_{be}
\end{math}

Der Ausgangswiderstand ist:

\begin{math}
  R_o = R_1 || R_{ce} = \frac{R_1 * R_{ce}}{R_1+R_{ce}}
\end{math}
