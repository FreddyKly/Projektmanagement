\section{Status zu Sprint 1
}\label{sec:status}

In diesem Sprint wurden bei allen Gruppenteilnehmern die IDE's und \LaTeX Umgebungen eingerichtet.
Für diesen Zweck nutzten alle die gleiche Plattform:
\begin{itemize}
  \item TexLive
  \item Visual Studio Code mit Latex-Workshop Plugin
  \item Git zur Versionierung
\end{itemize}
Auch wurde sich auf eine Ordnerstrucktur geeinigt. Statt Wochenberichte nutzten wir Sprintberichte die unsere Arbeitsweise wiederspiegeln.
Hierbei bildet jede Abgabeeinheit(P03..5) einen eigenen Sprint.
\begin{itemize}
  \item 20-p08.tex
  \item hda
  \item Wochenbericht
  \item Sprint (1..3)
  \begin{itemize}
    \item done.tex
    \item help.tex
    \item status.tex
    \item todo.tex
  \end{itemize}
  \item project.tex
  \item contact.tex
  \item contact-lecturer.tex
  \item contact-student.tex
\end{itemize}
Wir einigten uns auf ein Thema "Vergleich verschiedener Pegelwandler Topologien", da dieses eine gute Möglichkeit bietet z.B. Messwerte und Schaltungen zu visualisieren.
\begin{itemize}
  \item Spannungsteiler
  \item Transistor, invertierend
  \item Transistor, nicht invertierend
  \item Galvanisch getrennt
\end{itemize}
Im weiteren haben wir noch einen Plan aufgestellt wie die weiteren Sprints gestalltet werden um das Projekt möglichst effizient zu erarbeiten.

% EOF 