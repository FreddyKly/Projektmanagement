\subsection{Spannungsteiler
}\label{subsec:topo-spannungsteiler}

Die erste betrachtete Topologie stellt ein einfacher resistiver Spannungsteiler. 
Hier wird die Eingangsspannung durch 2 Widerstände so geteilt, dass das gewünschte Spannungsverhältnis am Ausgang anliegt.
Ein Problem dieser Schaltung ist, dass sie nur eine große in eine kleine Spannung wandeln kann.
\begin{figure}
  \begin{circuitikz}
    \draw (0,0)
      to[R=$R2$] (0,2)
      to[R=$R1$,-o] (0,4) node[right]{$input$};
    \draw (0,0) node[ground] {} (0,-1);
    \draw (0,2) to[short, -o] (1,2) node[right]{$output$};
  \end{circuitikz}
  \caption{Spannungsteiler}
  \label{fig:spannungsteiler}
\end{figure}

Die Ausgangsspannung wird dabei wie folgt bestimmt:

\begin{math}
  U_{output}(t) = U_{input}(t) * \frac{R_2}{R_1+R_2}
\end{math}

Der Einganswiderstand berechnet sich aus der Reihenschaltung der beiden Widerstände:

\begin{math}
  R_i = R_1+R_2
\end{math}

Der Ausgangswiderstand ist:

\begin{math}
  R_o = R_1 || R_2 = \frac{R_1 * R_2}{R_1+R_2}
\end{math}


% EOF