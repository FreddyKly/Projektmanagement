\section{Vergleich verschiedener Pegelwandlertechnologien 
}\label{sec:proj-title-long}

\subsection{Abstract
}\label{subsec:proj-abstract}

Pegelwandler sind in der Elektrotechnik weitreichend eingesetzte Schaltungsbauteile.
Vor allem im Bereich alter Schnittstellen oder z.B. bei SD Karten sind sie zu finden.
In diesem Projekt werden verschiedene Arten von Pegelwandlern beleuchtet und im Bezug ihrer Nützlichkeit bewertet.

\subsection{Definition
}\label{subsec:proj-def}

Die Pegelwandler sollen mittels Schaltungssimulation auf verschiedene Parameter untersucht werden.
Diese beinhalten u.A. 
\begin{itemize}
  \item Frequqnzverhalten
  \item Direktionalem Verhalten
  \item Komplexität
  \item Preis
\end{itemize}

\subsection{Ziel / Ergebnis
}\label{subsec:proj-target}

Am Ende soll eine fundierte Aussage zu dem verschiedenen Pegelwandlerarchitekturen zu geben.

% EOF
