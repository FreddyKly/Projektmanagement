\section{Vergleich verschiedener Pegelwandler Topologien 
}\label{sec:proj-title-long}

\subsection{Abstract
}\label{subsec:proj-abstract}

Pegelwandler sind in der Elektrotechnik weitreichend eingesetzte Schaltungselemente.
Vor allem im Bereich alter Schnittstellen oder z.B. bei SD Karten sind sie zu finden.
In diesem Projekt werden verschiedene Arten von Pegelwandlern beleuchtet und im Bezug ihrer Nützlichkeit bewertet.

\subsection{Definition
}\label{subsec:proj-def}

Die Pegelwandler sollen mittels Schaltungssimulation auf verschiedene Parameter untersucht werden.
Diese beinhalten u.A. 
\begin{itemize}
  \item Eingangswiderstand
  \item Ausgangswiderstand
  \item Frequenzverhalten
  \item Komplexität
  \item Direktionalem Verhalten
  \item Preis
\end{itemize}

Ein und Ausgangswiderstand wird rein theoretisch auf dem mathematischen Grundlagen der Schaltungen beschrieben.

Frequenzverhalten wird mittels LTSpice simuliert.

Komplexität und Preis stellen subjektive aussagen dar. Um eine vergleichbarkeit zu gewährleisten wurden jeweils die günstigsten passenden Bauelemente bei einem großen Distributor gewählt.



\subsection{Ziel / Ergebnis
}\label{subsec:proj-target}

Am Ende soll eine fundierte Aussage zu dem verschiedenen Pegelwandlerarchitekturen zu geben.

\newpage
\section{Betrachtete Pegelwandler Topologien}\label{sec:pegel-topo}

\newpage
\subsection{Spannungsteiler
}\label{subsec:topo-spannungsteiler}

Die erste betrachtete Topologie stellt ein einfacher resistiver Spannungsteiler. 
Hier wird die Eingangsspannung durch 2 Widerstände so geteilt, dass das gewünschte Spannungsverhältnis am Ausgang anliegt.
Ein Problem dieser Schaltung ist, dass sie nur eine große in eine kleine Spannung wandeln kann.
\begin{figure}
  \begin{circuitikz}
    \draw (0,0)
      to[R=$R2$] (0,2)
      to[R=$R1$,-o] (0,4) node[right]{$input$};
    \draw (0,0) node[ground] {} (0,-1);
    \draw (0,2) to[short, -o] (1,2) node[right]{$output$};
  \end{circuitikz}
  \caption{Spannungsteiler}
  \label{fig:spannungsteiler}
\end{figure}

Die Ausgangsspannung wird dabei wie folgt bestimmt:

\begin{math}
  U_{output}(t) = U_{input}(t) * \frac{R_2}{R_1+R_2}
\end{math}

Der Einganswiderstand berechnet sich aus der Reihenschaltung der beiden Widerstände:

\begin{math}
  R_i = R_1+R_2
\end{math}

Der Ausgangswiderstand ist:

\begin{math}
  R_o = R_1 || R_2 = \frac{R_1 * R_2}{R_1+R_2}
\end{math}


% EOF
\newpage
\subsection{Invertierender Transistor
}\label{subsec:topo-inv-transistor}

Diese Schaltung stellt eine Emitterschaltung dar. Sie nutzt einen NPN Transistor als Schalter um die Spannungswandlung vorzunehmen.
Ein Vorteil ist das sie die Spannung sowohl nieder als auch hoch setzten kann. Ein Nachteil ist, dass sie den Logik Pegel invertiert. 
\begin{figure}
  \begin{circuitikz}
    \draw (0,0) node[npn](Q1) {2N2222}
    (Q1.base) node[anchor=east] {}
    (Q1.collector) node[anchor=south] {}
    (Q1.emitter) node[anchor=north] {};
    \draw (Q1.collector) to[R, l2=$R_1$ and \SI {10}{k\ohm}] ++(0,2)
    node[vcc](){\SI{3.3}{V}};
    \draw (Q1.collector) to[short, -o] ++(2,0) node[right]{$output$};
    \draw (Q1.emitter) node[ground] {} ++(0,-1);
    \draw (Q1.base) to[R, -o, l2=$R_2$ and \SI {10}{k\ohm}] ++(-2,0)
    node[left]{$input$};
  \end{circuitikz}
  \caption{inv. Transistor}
  \label{fig:spannungsteiler}
\end{figure}

Der Einganswiderstand berechnet sich den Basiswiederstand und $R_{be}$:

\begin{math}
  R_i = R_2 + R_{be}
\end{math}

Der Ausgangswiderstand ist:

\begin{math}
  R_o = R_1 || R_{ce} = \frac{R_1 * R_{ce}}{R_1+R_{ce}}
\end{math}

\newpage
\subsection{MosFet Pegelwandler
}\label{subsec:topo-spannungsteiler}

Dieser Pegelwandler besitzt ein N Kanal Mosfet als Grundlegendes Bauelement.
Der große Vorteil dieser Schaltung liegt darin, dass sie bidirektional funktioniert.
Es wird der Weg von $IO_2$ zu $IO_1$ betrachtet.

\begin{figure}
  \begin{circuitikz}
    \draw (0,0) node[nigfete](Q1) {2N7002}
    (Q1.gate) node[anchor=east] {}
    (Q1.drain) node[anchor=south] {}
    (Q1.source) node[anchor=north] {};
    \draw (Q1.gate) to[short] ++(0,3)
    to[short] ++(1,0);
    \draw (Q1.drain) to[R, l2_=$R_1$ and \SI {10}{k\ohm}] ++(0,2)
    node[vcc](){\SI{3.3}{V}};
    \draw (Q1.source) to[R, l2_=$R_2$ and \SI {10}{k\ohm}] ++(0,-2)
    node[vee](){\SI{5}{V}};
    \draw (Q1.collector) to[short, -o] ++(2,0) node[right]{$IO_1$};
    \draw (Q1.source) to[short, -o] ++(2,0) node[right]{$IO_2$};
  \end{circuitikz}
  \caption{MosFet Bidirektional}
  \label{fig:mosfet}
\end{figure}


Der Einganswiderstand an $IO_2$ berechnet wie folgt:

Logisch 0:
\begin{math}
  R_i = R_1 || R_{DS} + R_2
\end{math}

Im Falle einer logischen 1 nahezu $\infty$

Der Ausgangswiderstand an $IO_1$ ist:

\begin{math}
  R_o = R_1
\end{math}




\newpage
\section{Vergleichstabelle}\label{sec:tab-vergleich}

\begin{table}
  
  \caption{Vergleichstabelle mit $U_{input} = \SI {}{V} $ und $U_{output} = \SI {}{V} $}
  \label{table:tabelle-vergleich1}
\end{table}
\begin{tabular}{l|l|l|l}
  & Spannungsteiler & inv. Transistor & MofFet Wandler \\
  \hline 
  Eingangswiderstand        & & & \\
  Ausgangswiderstand        & & & \\
  Frequenzverhalten         & & & \\
  Komplexität               & +++ & + & +  \\
  Direktionalem Verhalten   & & & \\
  Preis                     & & & \\
\end{tabular}




% EOF
