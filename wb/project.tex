\section{Vergleich verschiedener Pegelwandler Topologien 
}\label{sec:proj-title-long}

\subsection{Abstract
}\label{subsec:proj-abstract}

Pegelwandler sind in der Elektrotechnik weitreichend eingesetzte Schaltungselemente.
Vor allem im Bereich alter Schnittstellen oder z.B. bei SD Karten sind sie zu finden.
In diesem Projekt werden verschiedene Arten von Pegelwandlern beleuchtet und im Bezug ihrer Nützlichkeit bewertet.

\subsection{Definition
}\label{subsec:proj-def}

Die Pegelwandler sollen mittels Schaltungssimulation auf verschiedene Parameter untersucht werden.
Diese beinhalten u.A. 
\begin{itemize}
  \item Frequenzverhalten
  \item Direktionalem Verhalten
  \item Komplexität
  \item Preis
\end{itemize}

Die drei zu vergleichenden Pegelwandlertopologien sind:

\begin{figure}
  \begin{circuitikz}
    \draw (0,0)
      to[R=$R2$] (0,2)
      to[R=$R1$,-o] (0,4) node[right]{$input$};
    \draw (0,0) node[ground] {} (0,-1);
    \draw (0,2) to[short, -o] (1,2) node[right]{$output$};
  \end{circuitikz}
  \caption{Spannungsteiler}
  \label{fig:spannungsteiler}
\end{figure}


\begin{figure}
  \begin{circuitikz}
    \draw (0,0) node[npn](Q1) {2N2222}
    (Q1.base) node[anchor=east] {}
    (Q1.collector) node[anchor=south] {}
    (Q1.emitter) node[anchor=north] {};
    \draw (Q1.collector) to[R, l2=$R_1$ and \SI {10}{k\ohm}] ++(0,2)
    node[vcc](){\SI{3.3}{V}};
    \draw (Q1.collector) to[short, -o] ++(2,0) node[right]{$output$};
    \draw (Q1.emitter) node[ground] {} ++(0,-1);
    \draw (Q1.base) to[R, -o, l2=$R_2$ and \SI {10}{k\ohm}] ++(-2,0)
    node[left]{$input$};
  \end{circuitikz}
  \caption{inv. Transistor}
  \label{fig:spannungsteiler}
\end{figure}



\begin{figure}
  \begin{circuitikz}
    \draw (0,0) node[nigfete](Q1) {2N7002}
    (Q1.gate) node[anchor=east] {}
    (Q1.drain) node[anchor=south] {}
    (Q1.source) node[anchor=north] {};
    \draw (Q1.gate) to[short] ++(0,3)
    to[short] ++(1,0);
    \draw (Q1.drain) to[R, l2_=$R_1$ and \SI {10}{k\ohm}] ++(0,2)
    node[vcc](){\SI{3.3}{V}};
    \draw (Q1.source) to[R, l2_=$R_2$ and \SI {10}{k\ohm}] ++(0,-2)
    node[vee](){\SI{5}{V}};
    \draw (Q1.collector) to[short, -o] ++(2,0) node[right]{$IO_1$};
    \draw (Q1.source) to[short, -o] ++(2,0) node[right]{$IO_2$};
  \end{circuitikz}
  \caption{MosFet Bidirektional}
  \label{fig:mosfet}
\end{figure}







\subsection{Ziel / Ergebnis
}\label{subsec:proj-target}

Am Ende soll eine fundierte Aussage zu dem verschiedenen Pegelwandlerarchitekturen zu geben.

% EOF
