\subsection{MosFet Pegelwandler
}\label{subsec:topo-spannungsteiler}

Dieser Pegelwandler besitzt ein N Kanal Mosfet als Grundlegendes Bauelement.
Der große Vorteil dieser Schaltung liegt darin, dass sie bidirektional funktioniert.
Es wird der Weg von $IO_2$ zu $IO_1$ betrachtet.

\begin{figure}
  \begin{circuitikz}
    \draw (0,0) node[nigfete](Q1) {2N7002}
    (Q1.gate) node[anchor=east] {}
    (Q1.drain) node[anchor=south] {}
    (Q1.source) node[anchor=north] {};
    \draw (Q1.gate) to[short] ++(0,3)
    to[short] ++(1,0);
    \draw (Q1.drain) to[R, l2_=$R_1$ and \SI {10}{k\ohm}] ++(0,2)
    node[vcc](){\SI{3.3}{V}};
    \draw (Q1.source) to[R, l2_=$R_2$ and \SI {10}{k\ohm}] ++(0,-2)
    node[vee](){\SI{5}{V}};
    \draw (Q1.collector) to[short, -o] ++(2,0) node[right]{$IO_1$};
    \draw (Q1.source) to[short, -o] ++(2,0) node[right]{$IO_2$};
  \end{circuitikz}
  \caption{MosFet Bidirektional}
  \label{fig:mosfet}
\end{figure}


Der Einganswiderstand an $IO_2$ berechnet wie folgt:

Logisch 0:
\begin{math}
  R_i = R_1 || R_{DS} + R_2
\end{math}

Im Falle einer logischen 1 nahezu $\infty$

Der Ausgangswiderstand an $IO_1$ ist:

\begin{math}
  R_o = R_1
\end{math}

